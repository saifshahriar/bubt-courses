\documentclass[12pt]{article}

\usepackage{lmodern} % Better font scaling
\usepackage{anyfontsize} % Allows arbitrary font sizes
\usepackage{amsmath}  % For mathematical equations
\usepackage{graphicx} % For including images
\usepackage{cite}     % For citations
\usepackage{geometry} % Page margins and layout
\usepackage{fancyhdr} % For custom headers
\usepackage{hyperref} % For hyperlinks and clickable references
\usepackage{enumitem}  % For custom list formatting
\usepackage{gensymb}

% Page Layout
\geometry{a4paper, margin=2.5cm}

% Change body font size to 13pt
\fontsize{13}{15}\selectfont

% Header and footer
\pagestyle{fancy}
\fancyhf{} % Clear all header and footer fields
\fancyhead[R]{Page \thepage} % Page number in the right header
\fancyfoot{} % Remove page number from footer

% Custom lowercase Roman numeral format with brackets
\renewcommand{\labelenumi}{(\roman{enumi})}
\renewcommand{\thesection}{}

\begin{document}
\begin{center}
    {\LARGE \textbf{Physics I Formulas}} \\  % Large bold title
\end{center}

\section{Chapter 1: Simple Harmonic Motion}
%\subsection{Gas Constant}
\begin{enumerate}[label=\roman*.]
	\item Velocity,
		\[
		v = \frac{y}{t}
		\]
		Where,
		\[
		\begin{aligned}
			v &= \text{velocity (m/s)} \\
			y &= \text{displacement (m)} \\
			t &= \text{time (sec)}
		\end{aligned}
		\]

	\item Acceleration,
		\[
		a = \frac{v}{t}
		\]
		Where,
		\[
		\begin{aligned}
			a &= \text{acceleration } (\text{m/s}^2) \\
			v &= \text{velocity (m/s)} \\
			t &= \text{time (sec)}
		\end{aligned}
		\]


	\item Work,
		\[
			W = Fh = mgh
		\]
		Where,
		\begin{align*}
			m &= \text{mass (kg)} \\
			h &= \text{height (m)}
		\end{align*}

	\item Density,
		\[
			\rho = \frac{m}{V}
		\]
		Where,
		\begin{align*}
			\rho &= \text{density } (\text{kg/m}^3) \\
			m &= \text{mass (kg)} \\
			V &= \text{volume (m)}
		\end{align*}

	\item Time Period,
		\[
			T = \frac{t}{N}
		\]
		Where,
		\begin{align*}
			T &= \text{time period (s)} \\
			s &= \text{time (s)} \\
			N &= \text{number of oscillations}
		\end{align*}

	\item Momentum,
		\[
			p = mv
		\]
		Where,
		\begin{align*}
			m &= \text{mass (kg)} \\
			v &= \text{velocity (m/s)}
		\end{align*}

	\item Simple Harmonic Motion (SHM),
		\[
			y = A \sin (\omega t + \phi)
		\]
		Where,
		\begin{align*}
			y &= \text{displacement from the equilibrium position} \\
			A &= \text{amplitude} \\
			\omega &= \text{angular frequency} \\
			t &= \text{time}
		\end{align*}

	\item Force:
		\[
			F = ma
		\]
		Where,
		\begin{align*}
			m &= \text{mass (kg)} \\
			a &= \text{acceleration } (\text{m/s}^2)
		\end{align*}

	\item Velocity in SHM:
		\[
			v = \frac{dy}{dt} = - A \omega \sin(\omega t + \phi), \quad \text{if displacement is } y = A \cos(\omega t + \phi)
		\]
		\[
			v = \frac{dy}{dt} = A \omega \cos(\omega t + \phi), \quad \text{if displacement is } y = A \sin(\omega t + \phi)
		\]

\end{enumerate}

\section{Chapter 15: Calorimetry}
%\subsection{Newton's law of cooling}
%\begin{enumerate}[label=\roman*.]
%	\item Rate of loss of heat,
%		$$ - \frac{dQ}{dt} = A \cdot E \cdot f (\theta - \theta_{0}) $$
%		Where,
%		\begin{align*}
%			A &= \text{area of the exposed surface} \\
%			E &= \text{emissive power of the surface}
%		\end{align*}
%\end{enumerate}

%\subsection{Gas Constant}
\begin{enumerate}[label=\roman*.]
	\item
		\[
		\begin{aligned}
			& PV = rT \\
			 \text{or,} & \ r = \frac{PV}{T} \\
		\end{aligned}
		\]
		Where,
		\begin{align*}
			r &= \text{specific gas constant} \\
			P &= \text{pressure} \\
			V &= \text{volume} \\
			T &= \text{temperature}
		\end{align*}

	\item
		\[
		\begin{aligned}
			& PV = mrT \\
			\text{or,} \ & P = \frac{m}{V} \ rT = \rho rT \\
			\text{or,} \ & r = \frac{P}{\rho T}
		\end{aligned}
		\]
		Where,
		\begin{align*}
			m &= \text{mass} \\
			r &= \text{gas constant}
		\end{align*}

	\item Quantity of heat $Q$ supplied to the gas is,
		\begin{align*}
			Q &= m c_{v} \cdot (T_{2} - T_{1})\\
			  &= m c_{v} \cdot dT
		\end{align*}
		Where,
		\begin{align*}
			m &= \text{mass} \\
			c_{v} &= \text{specific heat (at constant volume)} \\
			T_{2} - T_{1} &= \text{rise in temperature}
		\end{align*}

	\item Relation between two specific heats,
		\[
		\begin{aligned}
			\frac{c_p}{c_v} = \gamma
		\end{aligned}
		\]
		Where,
		\begin{align*}
			c_p &= \text{specific heat capacity at constant pressure} \\
			c_v &= \text{specific heat capacity at constant volume} \\
			\gamma &= \text{gamma constant}
		\end{align*}

	\item
		\[
		\begin{aligned}
			c_p - c_v = \frac{r}{J} \ cal
		\end{aligned}
		\]
		Where,
		\begin{align*}
			c_p &= \text{specific heat capacity at constant pressure} \\
			c_v &= \text{specific heat capacity at constant volume} \\
			r &= \text{specific gas constant} \\
			J &= \text{mechanical equivalent of heat}
		\end{align*}

	\item Relation between pressure and volume after an adiabatic change,
		$$ P_1 V_1^\gamma = P_2 V_2^\gamma = constant $$ \\
		$P_1$, $V_1$ be pressure and volume and $P_2$, $V_2$ after an
			adiabatic change.

	\item Relation between volume and temperature in an adiabatic change,
		$$ T_1 V_1^{\gamma - 1} = T_2 V_2^{\gamma - 1} = constant $$
	\item Relation between pressure and temperature in an adiabatic change,
		\begin{align*}
			& T_1 P_1^\frac{1 - \gamma}{\gamma} = T_2 P_2^\frac{1 - \gamma}{\gamma} = constant \\
			\text{or,} \ & T_1^\gamma P_1^{1 - \gamma} = T_2^\gamma P_2^{1 - \gamma} = constant
		\end{align*}
\end{enumerate}

\section{Chapter 16: Kinetic Theory of Gases}
%\subsection{Expression for pressure exerted by a gas}
\begin{enumerate}[label=\roman*.]
	\item
		For ideal gas,
		\[
		\begin{aligned}
			PV = nRT
		\end{aligned}
		\]
		Where,
		\begin{align*}
			P &= \text{pressure} \\
			n &= \text{mole} \\
			T &= \text{temperature}
		\end{align*}

	\item
		r.m.s. velocity,
		\[
		\begin{aligned}
			c = \sqrt{\frac{3P}{\rho}} = \sqrt{\frac{3RT}{m}}
		\end{aligned}
		\]
		Where,
		\begin{align*}
			P &= \text{atmospheric pressure} \\
			\rho &= \text{density} \\
			R &= \text{universal gas constant} \\
			T &= \text{temperature} \\
			m &= \text{mass of a single molecule}
		\end{align*}

	\item
		relation between $c$ and $T$,
		\[
		\begin{aligned}
			& c \propto \sqrt{T} \\
			or, \ & \frac{c_1}{c_2} = \sqrt{\frac{T_1}{T_2}}
		\end{aligned}
		\]
		Where,
		\begin{align*}
			c_1 &= \text{initial state of}\ c \\
			c_2 &= \text{final state of}\ c \\
			T_1 &= \text{initial temperature} \\
			T_2 &= \text{final temperature}
		\end{align*}

	\item
		Kinetic Energy per particle,
		\[
		\begin{aligned}
			K.E. \rightarrow \frac{1}{2} mc^2 = \frac{3}{2}RT
		\end{aligned}
		\]
		Where,
		\begin{align*}
			m &= \text{mass of a single gas molecule} \\
			R &= \text{ideal gas constant} = 8.314 \ \text{J/mol} \cdot \text{K} \\
			T &= \text{absolute temperature of gas in Kelvin (K)}
		\end{align*}

	\item
		Average Kinetic Energy per particle,
		\[
		\begin{aligned}
			A.K.E. \rightarrow \frac{1}{2} mc^2 = \frac{3}{2}KT = \frac{3}{2}\frac{R}{N}T
		\end{aligned}
		\]
		Where,
		\begin{align*}
			m &= \text{mass of a single gas molecule} \\
			R &= \text{ideal gas constant} = 8.314 \ \text{J/mol} \cdot \text{K} \\
			N &= \text{number of particles (atoms or molecules) in a mole of gas} \\
			T &= \text{absolute temperature of gas in Kelvin (K)}
		\end{align*}

	\item
		Molecular Kinetic Energy per particle,
		\[
		\begin{aligned}
			M.K.E. \rightarrow \frac{1}{2} mc^2 = \frac{3}{2}KT = \frac{3}{2}\frac{R}{M}T
		\end{aligned}
		\]
		Where,
		\begin{align*}
			m &= \text{mass of a single gas molecule} \\
			R &= \text{ideal gas constant} = 8.314 \ \text{J/mol} \cdot \text{K} \\
			M &= \text{molar mass of the gas in kilograms per mole (kg/mol)} \\
			T &= \text{absolute temperature of gas in Kelvin (K)}
		\end{align*}
\end{enumerate}

\section{Chapter 20: Thermodynamics}

\begin{enumerate}[label=\roman*.]
    \item Comparison of temperature readings from various thermometers,
    \[
    \frac{C}{5} = \frac{F - 32}{9} = \frac{K - 273}{5}
    \]
    Where,
    \begin{align*}
        C &= \text{temperature in } ^{\circ} \text{Celsius} \\
        F &= \text{temperature in } ^{\circ} \text{Fahrenheit} \\
        K &= \text{temperature in Kelvin}
    \end{align*}

	\item Amount of work done,
		$$ W = JH = J \ (Q_{1} - Q_{2})$$
		Where,
		\begin{align*}
			W  &= \text{work done} \\[10pt]
			J  &= \text{mechanical equivalent of heat} \\
			   &= 4.186 \ \text{joules/cal} \\
			   &= 4.186 \times 10^7 \ \text{ergs/cal} \\
			   &= 4186 \ \text{joules/kcal} \\[10pt]
			H  &= \text{amount of heat transferred} \\[10pt]
			Q_{1} &= \text{heat absorbed from the hot reservoir} \\
			Q_{2} &= \text{heat rejected to the cold reservoir}
		\end{align*}

	\item Amount of heat supplied to a system,
		$$ dQ = dU + \frac{dW}{J} $$
		Where,
		\begin{align*}
			dQ &= \text{heat supplied} \\
			dU &= \text{change in internal energy} \\
			\frac{dW}{J} &= \text{heat equivalent of the work done}
		\end{align*}

%\subsection{Work and the PV (indicator) diagram for a gas}
%\begin{enumerate}[label=\roman*.]
%	\item Total work done by gas during its passage from A to B,
%		$$ W_{AB} = \int_{V_{1}}^{V_{2}} P \cdot dV = P (V_{2} - V_{1})$$
%		Where,
%		\begin{align*}
%			A &= \text{initial state of the gas before the expansion takes place} \\
%			B &= \text{final state of the gas after the expansion takes place} \\
%		\end{align*}
%\end{enumerate}

	\item Efficiency,
		$$ \eta = \frac{W}{Q} $$
		Where,
		\begin{align*}
			W &= \text{amount of work} \\
			Q &= \text{amount of heat}
		\end{align*}

	\item When $W$ in joules and $Q$ in calories,
		$$ \eta = \frac{W}{JQ} $$
		Where,
		\begin{align*}
			J  &= \text{mechanical equivalent of heat} \\
			   &= 4.186 \ \text{joules/cal} \\
			   &= 4.186 \times 10^7 \ \text{ergs/cal} \\
			   &= 4186 \ \text{joules/kcal}
		\end{align*}

	$ \eta = \frac{W}{Q} $ \rightarrow \ When $W$ and $Q$ both are in the same
		unit (both in joules or both in calories). \\
	$ \eta = \frac{W}{JQ} $ \rightarrow \ When $W$ is in \textbf{joules}, but
		$Q$ is in \textbf{calories}.

	\item Also in relation to heat and temperature,
		$$ \eta = 1 - \frac{Q_{2}}{Q_{1}} = 1 - \frac{T_{2}}{T_{1}} $$
		Where,
		\begin{align*}
			Q_{1} &= \text{heat absorbed from the hot reservoir} \\
			Q_{2} &= \text{heat rejected to the cold reservoir} \\
			T_{1} &= \text{absolute temperature of the hot reservoir (in Kelvin)} \\
			T_{2} &= \text{absolute temperature of the cold reservoir (in Kelvin)}
		\end{align*}
		The \textbf{hot} reservoir is the \textbf{source}, as it provides heat
		to the engine. The \textbf{cold} reservoir is the \textbf{sink}, as it
		absorbs the heat that the engine rejects. \\
		So, the temperature of \textbf{source} shall always be \textbf{higher}
		than the temperature of \textbf{sink}.

	\item So, the relation between absolute temperature and heat,
		$$ \frac{Q_{1}}{Q_{2}} = \frac{T_{1}}{T_{2}} $$

	\item Coefficient of performance,
		$$ E = \frac{Q_2}{W} = \frac{Q_2}{Q_1 - Q_2} $$
		Where,
		\begin{align*}
			W &= \text{amount of work} \\
			Q_{1} &= \text{heat absorbed from the hot reservoir} \\
			Q_{2} &= \text{heat rejected to the cold reservoir}
		\end{align*}
\end{enumerate}

\end{document}
