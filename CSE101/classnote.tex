% Options for packages loaded elsewhere
\PassOptionsToPackage{unicode}{hyperref}
\PassOptionsToPackage{hyphens}{url}
%
\documentclass[
]{article}
\usepackage{lmodern}
\usepackage{amssymb,amsmath}
\usepackage{ifxetex,ifluatex}
\ifnum 0\ifxetex 1\fi\ifluatex 1\fi=0 % if pdftex
  \usepackage[T1]{fontenc}
  \usepackage[utf8]{inputenc}
  \usepackage{textcomp} % provide euro and other symbols
\else % if luatex or xetex
  \usepackage{unicode-math}
  \defaultfontfeatures{Scale=MatchLowercase}
  \defaultfontfeatures[\rmfamily]{Ligatures=TeX,Scale=1}
\fi
% Use upquote if available, for straight quotes in verbatim environments
\IfFileExists{upquote.sty}{\usepackage{upquote}}{}
\IfFileExists{microtype.sty}{% use microtype if available
  \usepackage[]{microtype}
  \UseMicrotypeSet[protrusion]{basicmath} % disable protrusion for tt fonts
}{}
\makeatletter
\@ifundefined{KOMAClassName}{% if non-KOMA class
  \IfFileExists{parskip.sty}{%
    \usepackage{parskip}
  }{% else
    \setlength{\parindent}{0pt}
    \setlength{\parskip}{6pt plus 2pt minus 1pt}}
}{% if KOMA class
  \KOMAoptions{parskip=half}}
\makeatother
\usepackage{xcolor}
\IfFileExists{xurl.sty}{\usepackage{xurl}}{} % add URL line breaks if available
\IfFileExists{bookmark.sty}{\usepackage{bookmark}}{\usepackage{hyperref}}
\hypersetup{
  pdftitle={CSE 101: Structured Programming Language},
  pdfauthor={Saif Shahriar},
  hidelinks,
  pdfcreator={LaTeX via pandoc}}
\urlstyle{same} % disable monospaced font for URLs
\usepackage[margin=1in]{geometry}
\usepackage{color}
\usepackage{fancyvrb}
\newcommand{\VerbBar}{|}
\newcommand{\VERB}{\Verb[commandchars=\\\{\}]}
\DefineVerbatimEnvironment{Highlighting}{Verbatim}{commandchars=\\\{\}}
% Add ',fontsize=\small' for more characters per line
\usepackage{framed}
\definecolor{shadecolor}{RGB}{248,248,248}
\newenvironment{Shaded}{\begin{snugshade}}{\end{snugshade}}
\newcommand{\AlertTok}[1]{\textcolor[rgb]{0.94,0.16,0.16}{#1}}
\newcommand{\AnnotationTok}[1]{\textcolor[rgb]{0.56,0.35,0.01}{\textbf{\textit{#1}}}}
\newcommand{\AttributeTok}[1]{\textcolor[rgb]{0.77,0.63,0.00}{#1}}
\newcommand{\BaseNTok}[1]{\textcolor[rgb]{0.00,0.00,0.81}{#1}}
\newcommand{\BuiltInTok}[1]{#1}
\newcommand{\CharTok}[1]{\textcolor[rgb]{0.31,0.60,0.02}{#1}}
\newcommand{\CommentTok}[1]{\textcolor[rgb]{0.56,0.35,0.01}{\textit{#1}}}
\newcommand{\CommentVarTok}[1]{\textcolor[rgb]{0.56,0.35,0.01}{\textbf{\textit{#1}}}}
\newcommand{\ConstantTok}[1]{\textcolor[rgb]{0.00,0.00,0.00}{#1}}
\newcommand{\ControlFlowTok}[1]{\textcolor[rgb]{0.13,0.29,0.53}{\textbf{#1}}}
\newcommand{\DataTypeTok}[1]{\textcolor[rgb]{0.13,0.29,0.53}{#1}}
\newcommand{\DecValTok}[1]{\textcolor[rgb]{0.00,0.00,0.81}{#1}}
\newcommand{\DocumentationTok}[1]{\textcolor[rgb]{0.56,0.35,0.01}{\textbf{\textit{#1}}}}
\newcommand{\ErrorTok}[1]{\textcolor[rgb]{0.64,0.00,0.00}{\textbf{#1}}}
\newcommand{\ExtensionTok}[1]{#1}
\newcommand{\FloatTok}[1]{\textcolor[rgb]{0.00,0.00,0.81}{#1}}
\newcommand{\FunctionTok}[1]{\textcolor[rgb]{0.00,0.00,0.00}{#1}}
\newcommand{\ImportTok}[1]{#1}
\newcommand{\InformationTok}[1]{\textcolor[rgb]{0.56,0.35,0.01}{\textbf{\textit{#1}}}}
\newcommand{\KeywordTok}[1]{\textcolor[rgb]{0.13,0.29,0.53}{\textbf{#1}}}
\newcommand{\NormalTok}[1]{#1}
\newcommand{\OperatorTok}[1]{\textcolor[rgb]{0.81,0.36,0.00}{\textbf{#1}}}
\newcommand{\OtherTok}[1]{\textcolor[rgb]{0.56,0.35,0.01}{#1}}
\newcommand{\PreprocessorTok}[1]{\textcolor[rgb]{0.56,0.35,0.01}{\textit{#1}}}
\newcommand{\RegionMarkerTok}[1]{#1}
\newcommand{\SpecialCharTok}[1]{\textcolor[rgb]{0.00,0.00,0.00}{#1}}
\newcommand{\SpecialStringTok}[1]{\textcolor[rgb]{0.31,0.60,0.02}{#1}}
\newcommand{\StringTok}[1]{\textcolor[rgb]{0.31,0.60,0.02}{#1}}
\newcommand{\VariableTok}[1]{\textcolor[rgb]{0.00,0.00,0.00}{#1}}
\newcommand{\VerbatimStringTok}[1]{\textcolor[rgb]{0.31,0.60,0.02}{#1}}
\newcommand{\WarningTok}[1]{\textcolor[rgb]{0.56,0.35,0.01}{\textbf{\textit{#1}}}}
\usepackage{longtable,booktabs}
% Correct order of tables after \paragraph or \subparagraph
\usepackage{etoolbox}
\makeatletter
\patchcmd\longtable{\par}{\if@noskipsec\mbox{}\fi\par}{}{}
\makeatother
% Allow footnotes in longtable head/foot
\IfFileExists{footnotehyper.sty}{\usepackage{footnotehyper}}{\usepackage{footnote}}
\makesavenoteenv{longtable}
\usepackage{graphicx}
\makeatletter
\def\maxwidth{\ifdim\Gin@nat@width>\linewidth\linewidth\else\Gin@nat@width\fi}
\def\maxheight{\ifdim\Gin@nat@height>\textheight\textheight\else\Gin@nat@height\fi}
\makeatother
% Scale images if necessary, so that they will not overflow the page
% margins by default, and it is still possible to overwrite the defaults
% using explicit options in \includegraphics[width, height, ...]{}
\setkeys{Gin}{width=\maxwidth,height=\maxheight,keepaspectratio}
% Set default figure placement to htbp
\makeatletter
\def\fps@figure{htbp}
\makeatother
\setlength{\emergencystretch}{3em} % prevent overfull lines
\providecommand{\tightlist}{%
  \setlength{\itemsep}{0pt}\setlength{\parskip}{0pt}}
\setcounter{secnumdepth}{5}
\usepackage{fontspec}
\usepackage{lmodern}
%\usepackage{anyfontsize}
%\usepackage{amsmath}
%\usepackage{graphicx}
%\usepackage{cite}
\usepackage{geometry}
\usepackage{fancyhdr}
\usepackage{hyperref}
\usepackage{multicol}
\usepackage{enumitem}
%\usepackage{listings}

\geometry{a4paper, margin=1in}
%\fontsize{13}{15}\selectfont
\pagestyle{fancy}
\fancyhf{}
\fancyhead[R]{Page \thepage}
\fancyfoot{}
\hypersetup{colorlinks=true, linkcolor=blue, urlcolor=blue, citecolor=blue}

\setmonofont{JetBrainsMono Nerd Font}[Scale=0.9]


\title{CSE 101: Structured Programming Language}
\author{Saif Shahriar}
\date{2025-02-15}

\begin{document}
\maketitle

\tableofcontents

\newpage

\hypertarget{preface}{%
\section{Preface}\label{preface}}

C is a general-purpose, procedural programming language that has had a
significant influence on modern programming languages. Developed in the
early \textbf{1970s by Dennis Ritchie at Bell Labs}, C was designed to
be simple, efficient, and flexible, making it ideal for system
programming, including operating system development.

Key features of C include:

\begin{enumerate}
\def\labelenumi{\arabic{enumi}.}
\tightlist
\item
  \textbf{Efficiency:} C allows direct manipulation of hardware
  resources and memory, making it a powerful tool for
  performance-critical applications.
\item
  \textbf{Portability:} Code written in C can run on different machines
  with little to no changes, contributing to its widespread adoption.
\item
  \textbf{Structured Programming:} C emphasizes structured programming
  techniques, making code more understandable and maintainable.
\item
  \textbf{Standard Library:} The C Standard Library provides essential
  functions for tasks such as input/output (I/O), string manipulation,
  and memory allocation. Overall, C forms the foundation for many modern
  programming languages and is still widely used for system-level
  programming, embedded systems, and developing software that requires
  high performance.
\end{enumerate}

\hypertarget{components-of-a-c-program-your-first-c-program}{%
\section{Components of a C Program (Your First C
Program)}\label{components-of-a-c-program-your-first-c-program}}

Basic structure of a C program:

\begin{Shaded}
\begin{Highlighting}[]
\PreprocessorTok{\#include }\ImportTok{\textless{}stdio.h\textgreater{}}

\DataTypeTok{int}\NormalTok{ main() \{}
\NormalTok{    printf(}\StringTok{"Hello World!}\SpecialCharTok{\textbackslash{}n}\StringTok{"}\NormalTok{);}
    \ControlFlowTok{return} \DecValTok{0}\NormalTok{;}
\NormalTok{\}}
\end{Highlighting}
\end{Shaded}

Output:

\begin{verbatim}
## Hello World!
\end{verbatim}

\hypertarget{header-files}{%
\subsection{Header Files}\label{header-files}}

C is very modular. For example, the
\texttt{\#include\ \textless{}stdio.h\textgreater{}} directive tells the
compiler to include the contents of the \texttt{stdio.h} file in the
program. These \texttt{*.h} files are known as \textbf{header files}.
This file contains declarations for the standard input/output library
functions, such as \texttt{printf} and \texttt{scanf}. Here,
\texttt{\#include} just links the header file to the program.

\hypertarget{main-function}{%
\subsection{Main Function}\label{main-function}}

The \texttt{main()} function is the entry point of a C program. It is
mandatory to have a \texttt{main()} function in every C program. Any
other \protect\hyperlink{functions}{functions} can be.

\hypertarget{printf-function}{%
\subsection{\texorpdfstring{\texttt{printf()}
Function}{printf() Function}}\label{printf-function}}

Prints the output to the console.

\hypertarget{return-statement}{%
\subsection{Return Statement}\label{return-statement}}

\texttt{return\ 0;} is used to return a value from the \texttt{main()}
function. The value \texttt{0} represents the successful execution of
the program. Any other value indicates an error.

\hypertarget{c-tokenization}{%
\section{C Tokenization}\label{c-tokenization}}

There 5 types of tokens in C programming language.\footnote{Most places
  mentions 6 types of tokens. According to codeforwin, it it 5.}

\hypertarget{keywords}{%
\subsection{Keywords}\label{keywords}}

Reserved keywords C programming language. A total of 32 keywords in C as
per ANSI standards.

\begin{multicols}{4}
\begin{enumerate}
\item auto
\item break
\item case
\item char
\item const
\item continue
\item default
\item do
\item double
\item else
\item enum
\item extern
\item float
\item for
\item goto
\item if
\item int
\item long
\item register
\item return
\item short
\item signed
\item sizeof
\item static
\item struct
\item switch
\item typedef
\item union
\item unsigned
\item void
\item volatile
\item while
\end{enumerate}
\end{multicols}

\hypertarget{identifiers}{%
\subsection{Identifiers}\label{identifiers}}

Identifiers are the names used for things in programming, like
variables, functions, and custom types.

Rules for naming an identifier:

\begin{enumerate}
\def\labelenumi{\arabic{enumi}.}
\tightlist
\item
  Cannot be a \protect\hyperlink{keywords}{reserved keyword}.
\item
  Must begin with an alphabet \texttt{a-z}, \texttt{A-Z} or underscore
  \texttt{\_} symbol.
\item
  Can contain digits \texttt{0-9} but not at the beginning.
\item
  Must not contain any special character except underscore \texttt{\_}.
\item
  No white spaces are allowed in identifiers.
\item
  Identifiers are case-sensitive. So, \texttt{sum} and \texttt{SUM} are
  different.
\item
  There is no limit on the length of an identifier.
\end{enumerate}

Example:

\begin{longtable}[]{@{}ll@{}}
\toprule
\begin{minipage}[b]{0.53\columnwidth}\raggedright
Valied Identifiers\strut
\end{minipage} & \begin{minipage}[b]{0.41\columnwidth}\raggedright
Invalid Identifiers\strut
\end{minipage}\tabularnewline
\midrule
\endhead
\begin{minipage}[t]{0.53\columnwidth}\raggedright
\texttt{name}, \texttt{\_name}, \texttt{name123}, \texttt{name\_123},
\texttt{\_name123}, \texttt{NAME}\strut
\end{minipage} & \begin{minipage}[t]{0.41\columnwidth}\raggedright
\texttt{name\ 123}, \texttt{123name}, \texttt{name@123},
\texttt{name-123}\strut
\end{minipage}\tabularnewline
\bottomrule
\end{longtable}

\hypertarget{operators}{%
\subsection{Operators}\label{operators}}

\hypertarget{arithmetic-operators}.

Usage:

\begin{Shaded}
\begin{Highlighting}[]
\PreprocessorTok{\#include }\ImportTok{\textless{}stdio.h\textgreater{}}

\DataTypeTok{int}\NormalTok{ main() \{}
    \DataTypeTok{int}\NormalTok{ a = }\DecValTok{10}\NormalTok{, b = }\DecValTok{20}\NormalTok{;}
\NormalTok{    printf(}\StringTok{"a + b = \%d}\SpecialCharTok{\textbackslash{}n}\StringTok{"}\NormalTok{, a + b);}
    \ControlFlowTok{return} \DecValTok{0}\NormalTok{;}
\NormalTok{\}}
\end{Highlighting}
\end{Shaded}

Output:

\begin{verbatim}
## a + b = 30
\end{verbatim}

\hypertarget{assignment-operators}{%
\subsubsection{Assignment Operators}\label{assignment-operators}}

Assign values to variables.

Example: \texttt{=}, \texttt{+=}, \texttt{-=}, \texttt{*=}, \texttt{/=},
\texttt{\%=}, \texttt{\textless{}\textless{}=},
\texttt{\textgreater{}\textgreater{}=}, \texttt{\&=}, \texttt{\^{}=},
\texttt{\textbar{}=}.

Usage::

\begin{Shaded}
\begin{Highlighting}[]
\PreprocessorTok{\#include }\ImportTok{\textless{}stdio.h\textgreater{}}

\DataTypeTok{int}\NormalTok{ main() \{}
    \DataTypeTok{int}\NormalTok{ a = }\DecValTok{10}\NormalTok{, b = }\DecValTok{20}\NormalTok{;}
\NormalTok{    a += b;    }\CommentTok{// adds b to a and assigns the result to a}
\NormalTok{    printf(}\StringTok{"a = \%d}\SpecialCharTok{\textbackslash{}n}\StringTok{"}\NormalTok{, a);}
    \ControlFlowTok{return} \DecValTok{0}\NormalTok{;}
\NormalTok{\}}
\end{Highlighting}
\end{Shaded}

Output:

\begin{verbatim}
## a = 30
\end{verbatim}

\hypertarget{relational-operators}{%
\subsubsection{Relational Operators}\label{relational-operators}}

Relational Operators are used to compare two values.

Example: \texttt{==}, \texttt{!=}, \texttt{\textgreater{}},
\texttt{\textless{}}, \texttt{\textgreater{}=}, \texttt{\textless{}=}.

Usage:

\begin{Shaded}
\begin{Highlighting}[]
\PreprocessorTok{\#include }\ImportTok{\textless{}stdio.h\textgreater{}}

\DataTypeTok{int}\NormalTok{ main() \{}
    \DataTypeTok{int}\NormalTok{ a = }\DecValTok{10}\NormalTok{, b = }\DecValTok{20}\NormalTok{;}
    \ControlFlowTok{if}\NormalTok{ (a \textgreater{} b)}
\NormalTok{        printf(}\StringTok{"a is greater than b}\SpecialCharTok{\textbackslash{}n}\StringTok{"}\NormalTok{);}
    \ControlFlowTok{else}
\NormalTok{        printf(}\StringTok{"b is greater than a}\SpecialCharTok{\textbackslash{}n}\StringTok{"}\NormalTok{);}
    \ControlFlowTok{return} \DecValTok{0}\NormalTok{;}
\NormalTok{\}}
\end{Highlighting}
\end{Shaded}

Output:

\begin{verbatim}
## b is greater than a
\end{verbatim}

\hypertarget{logical-operators}{%
\subsubsection{Logical Operators}\label{logical-operators}}

Operators that perform logical operations.

Example: \texttt{\&\&}, \texttt{\textbar{}\textbar{}}, \texttt{!}.

Usage:

\begin{Shaded}
\begin{Highlighting}[]
\PreprocessorTok{\#include }\ImportTok{\textless{}stdio.h\textgreater{}}

\DataTypeTok{int}\NormalTok{ main() \{}
    \DataTypeTok{int}\NormalTok{ a = }\DecValTok{10}\NormalTok{, b = }\DecValTok{20}\NormalTok{;}
    \ControlFlowTok{if}\NormalTok{ (a \textgreater{} }\DecValTok{0}\NormalTok{ \&\& b \textgreater{} }\DecValTok{0}\NormalTok{)}
\NormalTok{        printf(}\StringTok{"Both a and b are positive}\SpecialCharTok{\textbackslash{}n}\StringTok{"}\NormalTok{);}
    \ControlFlowTok{return} \DecValTok{0}\NormalTok{;}
\NormalTok{\}}
\end{Highlighting}
\end{Shaded}

Output:

\begin{verbatim}
## Both a and b are positive
\end{verbatim}

\hypertarget{bitwise-operators}{%
\subsubsection{Bitwise Operators}\label{bitwise-operators}}

Direct manipulation of bits.

Example: \texttt{\&}, \texttt{\textbar{}}, \texttt{\^{}},
\texttt{\textasciitilde{}}, \texttt{\textless{}\textless{}},
\texttt{\textgreater{}\textgreater{}}.

Usage:

\begin{Shaded}
\begin{Highlighting}[]
\PreprocessorTok{\#include }\ImportTok{\textless{}stdio.h\textgreater{}}

\DataTypeTok{int}\NormalTok{ main() \{}
    \DataTypeTok{int}\NormalTok{ a = }\DecValTok{10}\NormalTok{, b = }\DecValTok{20}\NormalTok{;}
\NormalTok{    printf(}\StringTok{"a \& b = \%d}\SpecialCharTok{\textbackslash{}n}\StringTok{"}\NormalTok{, a \& b);}
    \ControlFlowTok{return} \DecValTok{0}\NormalTok{;}
\NormalTok{\}}
\end{Highlighting}
\end{Shaded}

Output:

\begin{verbatim}
## a & b = 0
\end{verbatim}

\hypertarget{incrementdecrement-operators}{%
\subsubsection{Increment/Decrement
Operators}\label{incrementdecrement-operators}}

Increment or decrement the value of a variable.

Example: \texttt{++}, \texttt{-\/-}.

Usage:

\begin{Shaded}
\begin{Highlighting}[]
\PreprocessorTok{\#include }\ImportTok{\textless{}stdio.h\textgreater{}}

\DataTypeTok{int}\NormalTok{ main() \{}
    \DataTypeTok{int}\NormalTok{ a = }\DecValTok{10}\NormalTok{;}
\NormalTok{    a++;}
\NormalTok{    printf(}\StringTok{"a = \%d}\SpecialCharTok{\textbackslash{}n}\StringTok{"}\NormalTok{, a);}
    \ControlFlowTok{return} \DecValTok{0}\NormalTok{;}
\NormalTok{\}}
\end{Highlighting}
\end{Shaded}

Output:

\begin{verbatim}
## a = 11
\end{verbatim}

\hypertarget{conditional-ternary-operator}{%
\subsubsection{Conditional (Ternary)
Operator}\label{conditional-ternary-operator}}

Operator that checks a condition and returns a value based on the
condition. A simple if-else statement in a single line.

Example: \texttt{condition\ ?\ value\_if\_true\ :\ value\_if\_false}.

Usage:

\begin{Shaded}
\begin{Highlighting}[]
\PreprocessorTok{\#include }\ImportTok{\textless{}stdio.h\textgreater{}}

\DataTypeTok{int}\NormalTok{ main() \{}
    \DataTypeTok{int}\NormalTok{ a = }\DecValTok{10}\NormalTok{, b = }\DecValTok{20}\NormalTok{;}
    \DataTypeTok{int}\NormalTok{ max = (a \textgreater{} b) ? a : b;}
\NormalTok{    printf(}\StringTok{"Max = \%d}\SpecialCharTok{\textbackslash{}n}\StringTok{"}\NormalTok{, max);}
    \ControlFlowTok{return} \DecValTok{0}\NormalTok{;}
\NormalTok{\}}
\end{Highlighting}
\end{Shaded}

Output:

\begin{verbatim}
## Max = 20
\end{verbatim}

\hypertarget{other-operators}{%
\subsubsection{Other operators}\label{other-operators}}

And many more\ldots{}

Example:

\begin{itemize}
\tightlist
\item
  \texttt{sizeof} operator returns the size of a variable or data type
  in bytes.
\item
  \texttt{\&} operator returns the address of a variable.
\item
  \texttt{*} operator is used to declare a pointer to a variable.
\item
  \texttt{-\textgreater{}}, \texttt{.} operators are used to access
  structure members using a pointer.
\end{itemize}

Usage:

\begin{Shaded}
\begin{Highlighting}[]
\PreprocessorTok{\#include }\ImportTok{\textless{}stdio.h\textgreater{}}

\KeywordTok{typedef} \KeywordTok{struct}\NormalTok{ \{}
    \DataTypeTok{int}\NormalTok{ x;}
    \DataTypeTok{int}\NormalTok{ y;}
\NormalTok{\} Point;}

\DataTypeTok{int}\NormalTok{ main() \{}
    \DataTypeTok{int}\NormalTok{ a = }\DecValTok{10}\NormalTok{;}
\NormalTok{    printf(}\StringTok{"Size of a int: \%d bytes}\SpecialCharTok{\textbackslash{}n}\StringTok{"}\NormalTok{, }\KeywordTok{sizeof}\NormalTok{ a);}
\NormalTok{    printf(}\StringTok{"Address of a: \%p}\SpecialCharTok{\textbackslash{}n}\StringTok{"}\NormalTok{, \&a);}

\NormalTok{    Point p = \{}\DecValTok{1}\NormalTok{, }\DecValTok{2}\NormalTok{\};}
\NormalTok{    printf(}\StringTok{"p.x = \%d}\SpecialCharTok{\textbackslash{}n}\StringTok{"}\NormalTok{, p.x);}
\NormalTok{    printf(}\StringTok{"p.y = \%d}\SpecialCharTok{\textbackslash{}n}\StringTok{"}\NormalTok{, p.y);}
    \ControlFlowTok{return} \DecValTok{0}\NormalTok{;}
\NormalTok{\}}
\end{Highlighting}
\end{Shaded}

Output:

\begin{verbatim}
## Size of a int: 4 bytes
## Address of a: 0x7ffc0b090afc
## p.x = 1
## p.y = 2
\end{verbatim}

\hypertarget{seperator}{%
\subsection{Seperator}\label{seperator}}

Used to separate statements.

Example: \texttt{} (space), \texttt{,} (comma), \texttt{;} (semicolon),
\texttt{\textbackslash{}t} (tab).

\hypertarget{literals}{%
\subsection{Literals}\label{literals}}

Constant values that are used within the program.

Four types of literals:

\begin{enumerate}
\def\labelenumi{\arabic{enumi}.}
\tightlist
\item
  Integer literals: \texttt{10}, \texttt{20}, \texttt{30}.
\item
  Floating-point or real literals: \texttt{3.14}, \texttt{2.718}.
\item
  Character literals: \texttt{\textquotesingle{}a\textquotesingle{}},
  \texttt{\textquotesingle{}b\textquotesingle{}},
  \texttt{\textquotesingle{}c\textquotesingle{}}.
\item
  String literals: \texttt{"Hello"}, \texttt{"World"},
  \texttt{"I\ love\ programming!"}.
\end{enumerate}

\hypertarget{comments}{%
\section{Comments}\label{comments}}

Non-executable statements that are ignored by the compiler. 2 types of
comments in C.

\begin{enumerate}
\def\labelenumi{\arabic{enumi}.}
\tightlist
\item
  Single-line comments: \texttt{//\ This\ is\ a\ single-line\ comment}
\item
  Multi-line comments:
\end{enumerate}

\begin{Shaded}
\begin{Highlighting}[]
\PreprocessorTok{\#include }\ImportTok{\textless{}stdio.h\textgreater{}}

\DataTypeTok{int}\NormalTok{ main() \{}
    \CommentTok{/*}
\CommentTok{        This is a}
\CommentTok{        multi{-}line comment}
\CommentTok{        and would be ignored}
\CommentTok{        by the compiler}
\CommentTok{    */}
\NormalTok{    printf(}\StringTok{"Hello World!}\SpecialCharTok{\textbackslash{}n}\StringTok{"}\NormalTok{);}
    \ControlFlowTok{return} \DecValTok{0}\NormalTok{;}
\NormalTok{\}}
\end{Highlighting}
\end{Shaded}

Output:

\begin{verbatim}
## Hello World!
\end{verbatim}

\hypertarget{functions}{%
\section{Functions}\label{functions}}

Coming soon\ldots{}

\end{document}
